\documentclass[a4paper]{article}
\usepackage{tikz}
\usepackage{palatino}
\usepackage{fullpage}
% \usepackage{CJK}
\usepackage{geometry}
\usepackage[colorlinks = false,
            linkcolor = red,
            ]{hyperref}
\usepackage[normalem]{ulem}

% set margins
\leftmargin=0.25in
\oddsidemargin=0.25in
\textwidth=6.0in
\topmargin=-0.25in
\textheight=10in

\raggedright

\def\justifying{%
  \rightskip=0pt
  \spaceskip=0pt
  \xspaceskip=0pt
  \relax
}

\newcommand{\ExternalLink}{%
    \tikz[x=1.2ex, y=1.2ex, baseline=-0.05ex]{% 
        \begin{scope}[x=1ex, y=1ex]
            \clip (-0.1,-0.1) 
                --++ (-0, 1.2) 
                --++ (0.6, 0) 
                --++ (0, -0.6) 
                --++ (0.6, 0) 
                --++ (0, -1);
            \path[draw, 
                line width = 0.5, 
                rounded corners=0.5] 
                (0,0) rectangle (1,1);
        \end{scope}
        \path[draw, line width = 0.5] (0.5, 0.5) 
            -- (1, 1);
        \path[draw, line width = 0.5] (0.6, 1) 
            -- (1, 1) -- (1, 0.6);
        }
    }

\thispagestyle{empty}

\newenvironment{changemargin}[2]{%
  \begin{list}{}{%
    \setlength{\topsep}{0pt}%
    \setlength{\leftmargin}{#1}%
    \setlength{\rightmargin}{#2}%
    \setlength{\listparindent}{\parindent}%
    \setlength{\itemindent}{\parindent}%
    \setlength{\parsep}{\parskip}%
  }%
  \item[]}{\end{list}
}

\newcommand{\lineover}{
	\begin{changemargin}{-0.05in}{-0.05in}
		\vspace*{-8pt}
		\hrulefill \\
		\vspace*{-2pt}
	\end{changemargin}
}

\newcommand{\header}[1]{
	\begin{changemargin}{-0.5in}{-0.5in}
		\scshape{\large \textbf{#1}}\\
  	\lineover
	\end{changemargin}
}

\newcommand{\thesis}[1]{
	\small\emph{#1}\normalsize
}

\newcommand{\contact}[4]{
	\begin{changemargin}{-0.5in}{-0.5in}
		\begin{center}
			{\LARGE \scshape \textbf{#1}}\\ \smallskip
			404 Westwood Plaza, EVI 468 – 90095, Los Angeles, CA, USA\\ \smallskip
			{\href{#2}{#2}} ~ \smallskip 
			{\href{#3}{#3}} ~
			{\href{#4}{Google Scholar}}\smallskip
			% {\href{#6}{#6}}\smallskip
		\end{center}
	\end{changemargin}
}

\newenvironment{body} {
	\vspace*{-16pt}
	\begin{changemargin}{-0.5in}{-0.5in}
  }	
	{\end{changemargin}
}	

\newcommand{\school}[4]{
	\textbf{#1} \hfill \emph{#2\\}
	#3\\ 
	#4\\
}

\begin{document}
\contact{Zhengrong Wang}{seanzw@ucla.edu}{https://seanzw.github.io}{https://scholar.google.com/citations?user=h\_GwGfQAAAAJ\&hl=en}

% % Bio
% \header{Bio}

% \begin{body}
% 	\vspace{14pt}
% 	\justifying
%     \setlength{\parindent}{20pt}%
	
% 	% \indent

% 	Computer architecture, cache, microarchitecture, network-on-chip, CPU, GPU,
% 	compiler, simulator, C/C++/Python.

% \end{body}

% \smallskip
% \smallskip

% Education
\header{Education}

\begin{body}
	\vspace{14pt}

% ------
	\textbf{University of California, Los Angeles}, \emph{Department of Computer Science} \hfill Los Angeles, USA \\
Ph.D. in Computer Science, Advisor: Tony Nowatzki \hfill \emph{Aug. 2018 - Nov. 2023}{} \\
\thesis{Dissertation: General, Flexible and Unified Near-Data Computing} \\

\vspace{6pt}

% -----
	\textbf{University of California, Los Angeles}, \emph{Department of Computer Science} \hfill Los Angeles, USA \\
Master of Science in Computer Science \hfill \emph{Sep. 2016 - Jul. 2018}{} \\
\thesis{Thesis: An LLVM-IR Datagraph-Based Simulator for Flexible Design Space Exploration over Accelerator Architectures} \\

\vspace{6pt}

% -----
	\textbf{Tsinghua University}, \emph{Department of Electronic Engineering} \hfill Beijing, China \\
Bachlor of Engineering in Electronic Engineering, GPA: 91/100 \hfill \emph{Aug. 2012 - Jul. 2016}{} \\
\thesis{Thesis: Optimizing Convolutional Neural Network on FPGA under Heterogeneous Computing Framework with OpenCL} \\
% Rank: 30/241\\
\vspace{6pt}

% % -----
% 	\textbf{ETH Z\"urich}, \emph{Department of Information Technology} \hfill Z\"urich, Switzerland \\
% 	Exchange Student, International Academic Program, GPA: 5.50/ 6.00 \hfill \emph{Sept. 2014 - Feb. 2015}{} \\

\end{body}

\smallskip
\smallskip

%%%%%%%%%%%%%%%%%%%%%%%%%%%%%%%%%%%%%%%%%%%%%%%%%%%%%%%%%%%%%%%%%%%%%%%%%%%%%%%%
% Skills
\header{Professional Skills}

\begin{body}
	\vspace{14pt}
% -----
	Mathematic: Familiar with calculus, linear algebra, probability theory, discrete mathematics, algorithms.\\
	\smallskip
% -----
  Research Areas: Computer architecture, compiler, cache, microarchitecture, network-on-chip, CPU, GPU.\\
	\smallskip
% -----
	Programming: Skilled at C/C++, Python, assembly, LLVM IR.\\
	\smallskip

\end{body}
\smallskip
\smallskip

%%%%%%%%%%%%%%%%%%%%%%%%%%%%%%%%%%%%%%%%%%%%%%%%%%%%%%%%%%%%%%%%%%%%%%%%%%%%%%%%
\header{Professional Experiences}

\begin{body}
	\vspace{14pt}
	\textbf{Nvidia Research} \hfill \emph{Jun. 2022 - Sep. 2022} \\
	Research Scientist, Mentor: Neal Crago, Manager: Steve Keckler
	\begin{itemize}
	\itemsep 0pt
	\item Examine memory bottleneck in GPU for key machine learning kernels.
	\item Build a prototype of an enhanced tensor memory accelerator (TMA).
	\item Evaluted with state-of-the-art point cloud applications.
	\end{itemize}
	\smallskip

\end{body}
\smallskip
\smallskip

%%%%%%%%%%%%%%%%%%%%%%%%%%%%%%%%%%%%%%%%%%%%%%%%%%%%%%%%%%%%%%%%%%%%%%%%%%%%%%%%
% Projects
\header{Open Source Projects \& Infrastructures}

\begin{body}
	\vspace{14pt}

% -----
	\textbf{Gem5-AVX} \href{https://github.com/seanzw/gem5-avx}{\ExternalLink} \hfill \emph{Jan. 2019 - Present}\\
	First Author \& Maintainer \\
	\begin{itemize}
	\itemsep 0pt
	\item Add AVX-512 support to gem5 simulator, extensively used in research.
	\item Faithfully model the microarchiecture of vectorized instructions, including microops.
	\item Detailed tutorials on how to support new instructions.
	\end{itemize}
	\smallskip

% -----
	\textbf{Stream-Specialized Near-Data Acceleration Framework} \href{https://github.com/PolyArch/gem-forge-framework}{\ExternalLink} \hfill \emph{Jan. 2018 - Present}\\
	First Author \& Maintainer \\
	\begin{itemize}
	\itemsep 0pt
	\item Full-stack implementation of stream-specialized near-data acceleration.
	\item Include LLVM-based compiler transformation and end-to-end simulation in gem5.
	\item Results published in ISCA' 19, HPCA' 21, HPCA '22, MICRO '23, ASPLOS '23.
	\end{itemize}
	\smallskip

% -----

\end{body}
\smallskip
\smallskip

%%%%%%%%%%%%%%%%%%%%%%%%%%%%%%%%%%%%%%%%%%%%%%%%%%%%%%%%%%%%%%%%%%%%%%%%%%%%%%%%%
% Skills
\header{Publication}

\begin{body}
	\vspace{14pt}
	% -----
    Affinity Alloc: Taming \sout{Not-So} Near-Data Computing \\
	\textbf{Zhengrong Wang}, Christopher Liu, Nathan Beckmann, Tony Nowatzki \\
	\emph{IEEE/ACM International Symposium on Microarchitecture (MICRO)}, 2023, Toronto, Canada.\\
	\vspace{6pt}
	% -----
	% -----
    \underline{Inf}inity Stream: Portable and Programmer-Friendly \underline{I}n-/\underline{N}ear-Memory \underline{F}usion \\
	\textbf{Zhengrong Wang}, Christopher Liu, Aman Arora, Lizy John, Tony Nowatzki \\
	\emph{ACM International Conference on Architectural Support for Programming Languages and Operating Systems (ASPLOS)}, 2023, Vancouver, Canada.\\
	\vspace{6pt}
	% -----
    Infinity Stream: Enabling Transparent and Automated In-Memory Computing \\
	\textbf{Zhengrong Wang}, Christopher Liu, Tony Nowatzki \\
    \emph{IEEE Computer Architecture Letters, Vol. 21, No. 2, 2022}.\\
	\vspace{6pt}
	% -----
	OverGen: Improving FPGA Usability through Domain-specific Overlay Generation\\
	Sihao Liu, Jian Weng, Dylan Kupsh, Atefeh Sohrabizadeh, \textbf{Zhengrong Wang}, Licheng Guo, Jiuyang Liu, Maxim Zhulin, Lucheng Zhang, Jason Cong, Tony Nowatzki \\
	\emph{IEEE/ACM International Symposium on Microarchitecture (MICRO)}, 2022, Chicago, USA.\\
	\textbf{Best Paper Runner-Up} \\
	\vspace{6pt}
	% -----
	Near-Stream Computing: General and Transparent Near-Cache Acceleration \\
	\textbf{Zhengrong Wang}, Jian Weng, Sihao Liu, Tony Nowatzki \\
	\emph{IEEE International Symposium on High-Performance Computer Architecture (HPCA)}, 2022, Seoul, South Koera.\\
	\vspace{6pt}
	% -----
	Stream Floating: Enabling Proactive and Decentralized Cache Optimizations \\
	\textbf{Zhengrong Wang}, Jian Weng, Jason Lowe-Power, Jayesh Gaur, Tony Nowatzki \\
	\emph{IEEE International Symposium on High-Performance Computer Architecture (HPCA)}, 2021, Seoul, South Koera.\\
	\textbf{Best Paper Runner-Up} \\
	\vspace{6pt}
	% -----
	DSAGEN: Synthesizing Programmable Spatial Accelerators \\
	Jian Weng, Sihao Liu, Vidushi Dadu, \textbf{Zhengrong Wang}, Preyas Shah, Tony Nowatzki \\
	\emph{ACM International Symposium on Computer Architecture (ISCA)}, 2020, virtual.\\
	\textbf{IEEE Micro Top Picks Honorable Mention}\\
	\vspace{6pt}
	% -----
	A Hybrid Systolic-Dataflow Architecture for Inductive Matrix Algorithms\\
	Jian Weng, Sihao Liu, \textbf{Zhengrong Wang}, Vidush Dadu, Tony Nowatzki \\
	\emph{IEEE International Symposium on High-Performance Computer Architecture (HPCA)}, 2020, San Diego, USA.\\
	\vspace{6pt}
	% -----
	Stream-Based Memory Access Specialization for General Purpose Processors\\
	\textbf{Zhengrong Wang}, Tony Nowatzki \\
	\emph{ACM International Symposium on Computer Architecture (ISCA)}, 2019, Phoenix, USA.\\
	\vspace{6pt}
	% -----
	The Gem5 Simulator: Version 20.0+\\
	Jason Lowe-Power, Abdul Mutaal Ahmad, Ayaz Akram, ..., \textbf{Zhengrong Wang}, et al. \\
	\emph{arXiv:2007.03152v2}, 2020.\\
	\vspace{6pt}
	% -----
	Optimizing Convolutional Neural Network on FPGA under Heterogeneous Computing Framework with OpenCL\\
	\textbf{Zhengrong Wang}, Fei Qiao, Zhen Liu, Yuxiang Shan, Xunyi Zhou, Li Luo, Huazhong Yang \\
	\emph{IEEE Region 10 Conference (TENCON)}, 2016, Singapore.\\
\end{body}

\smallskip
\smallskip

%%%%%%%%%%%%%%%%%%%%%%%%%%%%%%%%%%%%%%%%%%%%%%%%%%%%%%%%%%%%%%%%%%%%%%%%%%%%%%%%
% Awards and Honors

% \newpage
\header{Awards and Honors}

\begin{body}
	\vspace{14pt}
	% -----
	Dissertation Year Fellowship, \emph{UCLA} \hfill{} \emph{June. 2023}\\
	\smallskip
	% -----
	Best Paper Runner-Up (OverGen, in \emph{MICRO '22}), \emph{IEEE} \hfill{} \emph{Oct. 2022}\\
	\smallskip
	% -----
	Best Paper Runner-Up (Stream Floating, in \emph{HPCA '21}), \emph{IEEE} \hfill{} \emph{Feb. 2021}\\
	\smallskip
	% -----
	IEEE Micro Top Picks 2020 Honorable Mention (DSAGEN, in \emph{ISCA '20}), \emph{IEEE} \hfill{} \emph{Jan. 2021}\\
	\smallskip
	% -----
	2021 Dongguan Entrepreneur Scholarship, \emph{Dongguan Entrepreneurs Federation} \hfill{} \emph{Nov. 2021}\\
	\smallskip
	% -----
	Second-class Scholarship for Excellent Freshmen, \emph{Tsinghua University} \hfill{} \emph{Oct. 2012}\\
	\smallskip
	% -----
	Wang Zhaosheng Scholarship for Excellent Student from Dongguan, \emph{Wang Zhaosheng Foundation} \hfill{} \emph{Oct. 2012}\\
	% \smallskip
	% % -----
	% Second Prize in $30^{\mathrm{th}}$ Chinese National Physics Contest (Non-Physical Group A) \hfill{} \emph{Dec. 2013}\\
	% \smallskip
	% % -----
	% Ranked No.5 in National Matriculation Test(Science), Guangdong Province (5/600,000) \hfill{} \emph{Jun. 2012}
\end{body}

\smallskip
\smallskip

% %%%%%%%%%%%%%%%%%%%%%%%%%%%%%%%%%%%%%%%%%%%%%%%%%%%%%%%%%%%%%%%%%%%%%%%%%%%%%%%%
% \newpage
% \header{Teaching}

% \begin{body}
% 	\vspace{14pt}
% 	\textbf{CS33: Introduction to Computer Organization} \hfill \emph{Mar. 2022 - June. 2022} \\
% 	Teaching Assistant w/ Prof. Glenn Reinman
% 	\begin{itemize}
% 	\itemsep 0pt
% 	\item Lead two-hour discussion every week, lab grading.
% 	\item Overall evaluation: 8/9.
% 	\item ``Very knowledgeable, helpful, and encouraging. He ensured that I understood the content."
% 	\end{itemize}
% 	\smallskip

% \end{body}
% \smallskip
% \smallskip

% %%%%%%%%%%%%%%%%%%%%%%%%%%%%%%%%%%%%%%%%%%%%%%%%%%%%%%%%%%%%%%%%%%%%%%%%%%%%%%%%
% \header{Research Mentoring}

% \begin{body}
% 	\vspace{14pt}
% 	\textbf{Christopher Liu} \href{https://chrisliu.org/}{\ExternalLink} \hfill \emph{CS PhD Student, UCLA} \\
% 	Develop the compiler support for Infinity Stream (Published in ASPLOS '23)	\hfill \emph{Dec. 2021 - Present} \\
% 	\smallskip
% 	\textbf{Nicholas Johnson} \hfill \emph{CS Undergraduate, UCLA} \\
% 	Improve stream support for high-performance join. \hfill \emph{Jun. 2023 - Sep. 2023} \\
% 	\smallskip
% 	\textbf{Arteen Abrishami} \href{arteen1000.github.io}{\ExternalLink} \hfill \emph{CS Undergraduate, UCLA} \\
% 	Explore near-data acceleration on chiplet architectures. \hfill \emph{Jan. 2023 - Now} \\
% 	\smallskip
% 	\textbf{Shyandeep Das} \hfill \emph{EE Master, UCLA} \\
% 	Support reuse for near-stream computing. Now Graduate Research Architect at ARM. \hfill \emph{Jan. 2023 - Jun. 2023} \\

% \end{body}
% \smallskip
% \smallskip

% %%%%%%%%%%%%%%%%%%%%%%%%%%%%%%%%%%%%%%%%%%%%%%%%%%%%%%%%%%%%%%%%%%%%%%%%%%%%%%%%
% \header{Community Services}

% \begin{body}
% 	\vspace{14pt}
% 	\textbf{IEEE Transactions on Computers} Reviewer \hfill \emph{Jul. 2023} \\
% 	\smallskip
% 	\textbf{ACM Transactions on Architecture and Code Optimization} Reviewer \hfill \emph{Jul. 2023} \\
% 	\smallskip
% 	\textbf{IEEE Transactions on Cloud Computing} Reviewer \hfill \emph{Oct. 2023} \\
% 	\smallskip
% 	\textbf{ISCA '23 uArch Workshop} \hfill \emph{Jun. 2023} \\
% 	Mentor for 4 talented undergraduate computer architects: \\
% 	Shujuan Chen (UC Davis), Viansa Schmulbach (UC Berkeley), \\
% 	Eden Alem (Washington University in St. Louis), and Evan Cheng (Stanford). \\
% 	\smallskip
% 	\textbf{HPCA '21} Student Reviewer \hfill \emph{Aug. 2020}\\

% \end{body}
% \smallskip
% \smallskip


%%%%%%%%%%%%%%%%%%%%%%%%%%%%%%%%%%%%%%%%%%%%%%%%%%%%%%%%%%%%%%%%%%%%%%%%%%%%%%%%
% Projects
% \header{Selected Projects \& Internships}

% \begin{body}
% 	\vspace{14pt}

% 	Yart-cpp, yet another ray tracer in C++ \hfill \emph{Sep. 2015 - Mar. 2016}
% 	\begin{itemize}
% 	\itemsep 0pt
% 	\item Implement direct lighting and bidirectional path tracing.
% 	\item Lambertian, specular, refraction and Cook-Torrance BRDF are supported.
% 	\item Use OC-Tree to accelerate complex mesh.
% 	\item Repo: \href{https://github.com/seanzw/yart-cpp}{https://github.com/seanzw/yart-cpp}
% 	\end{itemize}
% 	\smallskip

% % -----
% % -----

% 	Light C-Compiler in C\# \hfill \emph{Dec. 2015 - Mar. 2016}
% 	\begin{itemize}
% 	\itemsep 0pt
% 	\item Support C99 standard and generate x86 assembly.
% 	\item Handwritten regular expression, NFA scanner generator and parser combinator library.
% 	\item Scanner generated by handwritten scanner generator.
% 	\item Parser generated by handwritten parser combinator library.
% 	\item Semantic analysis that checks types and symbols;
% 	\item Code generation to x86 assembly on Windows (can be directly used by clang assembler)
% 	\item Repo: \href{https://github.com/seanzw/lcc}{https://github.com/seanzw/lcc}
% 	\end{itemize}
% 	\smallskip

% -----
% -----

	% OpenCL@FPGA (Undergraduate Thesis) \hfill \emph{Sep. 2015 - Jun. 2016}
	% \begin{itemize}
	% \itemsep 0pt
	% \item Supervised by Assoc. Prof. Fei Qiao, Tsinghua University
	% \item Use OpenCL to implement CNN on Xilinx Alpha Data FPGA, and accelerate with pipeline.
	% \item Paper on TENCON 16: Optimizing Convolutional Neural Network on FPGA under Heterogeneous Computing Framework with OpenCL
	% \end{itemize}
	% \smallskip

% \end{body}
% \smallskip
% \smallskip




% %%%%%%%%%%%%%%%%%%%%%%%%%%%%%%%%%%%%%%%%%%%%%%%%%%%%%%%%%%%%%%%%%%%%%%%%%%%%%%%%
% % Major Courses
% % \header{Main Courses}

% % \begin{body}
% % 	\vspace{14pt}

% % 	\begin{itemize} \itemsep -0pt
% % 		\item Mining Massive Datasets (Stanford CS246, Coursera)
% % 		\item Computer Program Design -- 98, 97 (2012 Fall \& 2013 Spring)
% % 		\item Data Structure, Numerical Analysis and Algorithms -- 89 (2013 Fall)
% % 		\item Probability and Stochastic Processes(1) -- 100 (2014 Spring)
% % 		\item Probability and Stochastic Processes(2) -- A (UW-Madison ECE 730, 2014 Fall)
% % 	\end{itemize}

% % \end{body}

% % \smallskip
% % \smallskip

% %%%%%%%%%%%%%%%%%%%%%%%%%%%%%%%%%%%%%%%%%%%%%%%%%%%%%%%%%%%%%%%%%%%%%%%%%%%%%%%%
% % Experience
% \header{Experience}

% \begin{body}
% 	\vspace{14pt}
% 	% -----
% 	\textbf{Courses in CS}
% 	\begin{itemize}
% 	\itemsep 0pt
% 	\item Compilers by Alex Aiken, Stanford University
% 	\item Operating System Engineering, MIT
% 	\item Programming Languages by Dan Grossman, University of Washington
% 	\item Machine Learning by Andrew Ng, Stanford University
% 	\item Algorithms Part I \& II by Robert Sedgewick, Princeton University
% 	\item Introduction to Computer Science and Programming, MIT
% 	\item Introduction to Probability, MIT
% 	\item Advanced Computer Graphics, Tsinghua University
% 	\item Computer Networks, Tsinghua University
% 	\item Software Engineering, Tsinghua University
% 	\item Computer Graphics (5.25/6), ETH Zurich
% 	\item Computer Vision (5.5/6), ETH Zurich
% 	\end{itemize}
% 	\smallskip

% 	% -----
% 	Children Education Program Volunteer, \emph{Dream a Dream, Bangalore, India} \hfill \emph{Jul. 2013 - Sept. 2013}\\
% 	\smallskip

% \end{body}

\smallskip
\smallskip



\end{document}