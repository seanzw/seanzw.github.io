\documentclass[a4paper]{article}
\usepackage{palatino}
\usepackage{fullpage}
% \usepackage{CJK}
\usepackage{geometry}
\usepackage[colorlinks = true,
            linkcolor = red,
            ]{hyperref}

% set margins
\leftmargin=0.25in
\oddsidemargin=0.25in
\textwidth=6.0in
\topmargin=0in
\textheight=9.25in

\raggedright

\thispagestyle{empty}


\newenvironment{changemargin}[2]{%
  \begin{list}{}{%
    \setlength{\topsep}{0pt}%
    \setlength{\leftmargin}{#1}%
    \setlength{\rightmargin}{#2}%
    \setlength{\listparindent}{\parindent}%
    \setlength{\itemindent}{\parindent}%
    \setlength{\parsep}{\parskip}%
  }%
  \item[]}{\end{list}
}

\newcommand{\lineover}{
	\begin{changemargin}{-0.05in}{-0.05in}
		\vspace*{-8pt}
		\hrulefill \\
		\vspace*{-2pt}
	\end{changemargin}
}

\newcommand{\header}[1]{
	\begin{changemargin}{-0.5in}{-0.5in}
		\scshape{\textbf{#1}}\\
  	\lineover
	\end{changemargin}
}

\newcommand{\contact}[4]{
	\begin{changemargin}{-0.5in}{-0.5in}
		\begin{center}
			{\Large \scshape {#1}}\\ \smallskip
			{\href{#2}{#2}} \\ \smallskip 
			{\href{#3}{Website}} /
			{\href{#4}{Google Scholar}}\smallskip
			% {\href{#6}{#6}}\smallskip
		\end{center}
	\end{changemargin}
}

\newenvironment{body} {
	\vspace*{-16pt}
	\begin{changemargin}{-0.25in}{-0.5in}
  }	
	{\end{changemargin}
}	

\newcommand{\school}[4]{
	\textbf{#1} \hfill \emph{#2\\}
	#3\\ 
	#4\\
}

\begin{document}
\contact{Zhengrong Wang}{seanzw@ucla.edu}{https://seanzw.github.io}{https://scholar.google.com/citations?user=h\_GwGfQAAAAJ\&hl=en}


% Education
\header{Education}

\begin{body}
	\vspace{14pt}

% ------
	\textbf{University of California, Los Angeles}, \emph{Department of Computer Science} \hfill Los Angeles, U.S. \\
Ph.D. in Computer Science \hfill \emph{Aug. 2018 - Jul. 2024} (Expected){} \\

% Rank: 30/241\\
\vspace{6pt}

% -----
	\textbf{University of California, Los Angeles}, \emph{Department of Computer Science} \hfill Los Angeles, U.S. \\
Master of Science in Computer Science \hfill \emph{Sep. 2016 - Jul. 2018}{} \\

% Rank: 30/241\\
\vspace{6pt}

% -----
	\textbf{Tsinghua University}, \emph{Department of Electronic Engineering} \hfill Beijing, P.R. China \\
Bachlor of Engineering in Electronic Engineering, GPA: 91/100 \hfill \emph{Aug. 2012 - Jul. 2016}{} \\
% Rank: 30/241\\
\vspace{6pt}

% -----
	\textbf{ETH Z\"urich}, \emph{Department of Information Technology} \hfill Z\"urich, Switzerland \\
	Exchange Student, International Academic Program, GPA: 5.50/ 6.00 \hfill \emph{Sept. 2014 - Feb. 2015}{} \\


\end{body}

\smallskip
\smallskip

%%%%%%%%%%%%%%%%%%%%%%%%%%%%%%%%%%%%%%%%%%%%%%%%%%%%%%%%%%%%%%%%%%%%%%%%%%%%%%%%%
% Skills
\header{Publication}

\begin{body}
	\vspace{14pt}
	% -----
	\textbf{Z. Wang}, J. Weng, S. Liu, T. Nowatzki. \\
	Near-Stream Computing: General and Transparent Near-Cache Acceleration. To Appear in \emph{HPCA '22}.\\
	\smallskip
	% -----
	\textbf{Best Paper Runner-Up: Z. Wang}, J. Weng, J. Lowe-Power, J. Gaur, T. Nowatzki. \\
	Stream Floating: Enabling Proactive and Decentralized Cache Optimizations. In \emph{HPCA '21}.\\
	\smallskip
	% -----
	\textbf{Z. Wang}, T. Nowatzki. \\
	Stream-Based Memory Access Specialization for General Purpose Processors. In \emph{ISCA '19}.\\
	\smallskip
	% -----
	\textbf{IEEE Micro Top Picks Honorable Mention} J. Weng*, S. Liu*, V. Dadu, \textbf{Z. Wang}, P. Shah, T. Nowatzki. \\
	DSAGEN: Synthesizing Programmable Spatial Accelerators. In \emph{ISCA '20}.\\
	\smallskip
	% -----
	J. Weng, S. Liu, \textbf{Z. Wang}, V. Dadu, T. Nowatzki. \\
	A Hybrid Systolic-Dataflow Architecture for Inductive Matrix Algorithms. In \emph{HPCA '20}.\\
	\smallskip
	% -----
	J. Lowe-Power, ..., \textbf{Z. Wang}, et al. \\
	The gem5 Simulator: Version 20.0+. In \emph{arXiv:2007.03152v2}.\\
	\smallskip
	% -----
	\textbf{Z. Wang}, F. Qiao, Z. Liu, Y. Shan, X. Zhou, L. Luo, and H. Zhong. \\
	Optimizing Convolutional Neural Network on FPGA under Heterogeneous Computing Framework with OpenCL. In \emph{TENCON '16}.\\
\end{body}

\smallskip
\smallskip

%%%%%%%%%%%%%%%%%%%%%%%%%%%%%%%%%%%%%%%%%%%%%%%%%%%%%%%%%%%%%%%%%%%%%%%%%%%%%%%%
% Awards and Honors

% \newpage
\header{Awards and Honors}

\begin{body}
	\vspace{14pt}
	% -----
	Best Paper Runner-Up, \emph{HPCA '21} \hfill{} \emph{Feb. 2021}\\
	\smallskip
	% -----
	IEEE Micro Top Picks Honorable Mention (DSAGen, ISCA '20), \emph{IEEE} \hfill{} \emph{Jan. 2021}\\
	\smallskip
	% -----
	Second-class Scholarship for Excellent Freshmen, \emph{Tsinghua University} \hfill{} \emph{Oct. 2012}\\
	\smallskip
	% -----
	Wang Zhaosheng Scholarship for Excellent Studeng from Dongguan, \emph{Wang Zhaosheng Fundation} \hfill{} \emph{Oct. 2012}\\
	\smallskip
	% -----
	Second Prize in $30^{\mathrm{th}}$ Chinese National Physics Contest(non-physical group A) \hfill{} \emph{Dec. 2013}\\
	\smallskip
	% -----
	Ranked No.5 in National Matriculation Test(Science), Guangdong Province (5/600,000) \hfill{} \emph{Jun. 2012}
\end{body}

\smallskip
\smallskip

%%%%%%%%%%%%%%%%%%%%%%%%%%%%%%%%%%%%%%%%%%%%%%%%%%%%%%%%%%%%%%%%%%%%%%%%%%%%%%%%
% Projects
\header{Selected Projects \& Internships}

\begin{body}
	\vspace{14pt}

% -----
	GemForge Framework \hfill \emph{Jan. 2018 - Present}
	\begin{itemize}
	\itemsep 0pt
	\item Research project of full-stack trace-based simulation for stream-specialized systems.
	\item Implement LLVM passes to recognize streams and transform program with new stream instructions.
	\item End-to-End execution-based simulation in gem5.
	\item Results published in ISCA' 19 and HPCA' 21. More in submission.
	\item Repo: \href{https://github.com/PolyArch/gem-forge-framework}{https://github.com/PolyArch/gem-forge-framework}
	\end{itemize}
	\smallskip

% -----
	Gem5-AVX \hfill \emph{Jan. 2019 - Present}
	\begin{itemize}
	\itemsep 0pt
	\item Add AVX-512 support to gem5 simulator, extensively used in research.
	\item Faithfully model the microarchiecture of vectorized instructions, including microops.
	\item Detailed tutorials on how to support new instructions.
	\item Repo: \href{https://github.com/seanzw/gem5-avx}{https://github.com/seanzw/gem5-avx}
	\end{itemize}
	\smallskip
% -----

% 	Yart-cpp, yet another ray tracer in C++ \hfill \emph{Sep. 2015 - Mar. 2016}
% 	\begin{itemize}
% 	\itemsep 0pt
% 	\item Implement direct lighting and bidirectional path tracing.
% 	\item Lambertian, specular, refraction and Cook-Torrance BRDF are supported.
% 	\item Use OC-Tree to accelerate complex mesh.
% 	\item Repo: \href{https://github.com/seanzw/yart-cpp}{https://github.com/seanzw/yart-cpp}
% 	\end{itemize}
% 	\smallskip

% % -----
% % -----

% 	Light C-Compiler in C\# \hfill \emph{Dec. 2015 - Mar. 2016}
% 	\begin{itemize}
% 	\itemsep 0pt
% 	\item Support C99 standard and generate x86 assembly.
% 	\item Handwritten regular expression, NFA scanner generator and parser combinator library.
% 	\item Scanner generated by handwritten scanner generator.
% 	\item Parser generated by handwritten parser combinator library.
% 	\item Semantic analysis that checks types and symbols;
% 	\item Code generation to x86 assembly on Windows (can be directly used by clang assembler)
% 	\item Repo: \href{https://github.com/seanzw/lcc}{https://github.com/seanzw/lcc}
% 	\end{itemize}
% 	\smallskip

% -----

	OpenCL@FPGA (Undergraduate Thesis) \hfill \emph{Sep. 2015 - Jun. 2016}
	\begin{itemize}
	\itemsep 0pt
	\item Supervised by Assoc. Prof. Fei Qiao, Tsinghua University
	\item Use OpenCL to implement CNN on Xilinx Alpha Data FPGA, and accelerate with pipeline.
	\item Paper on TENCON 16: Optimizing Convolutional Neural Network on FPGA under Heterogeneous Computing Framework with OpenCL
	\end{itemize}
	\smallskip

% -----
% -----

	MicroPython on FPGA, Dept. EEE, Imperial College London \hfill \emph{Jul. 2015 - Aug. 2015}
	\begin{itemize}
	\itemsep 0pt
	\item Supervised by Prof. Peter Y. K. Cheung, Head of Dept. EEE.
	\item Port MicroPython on Altera DE0-Nano-SoC FPGA.
	\item Build FFT example with DMA.
	\item Repo: \href{https://github.com/seanzw/MicroPythonFPGA}{https://github.com/seanzw/MicroPythonFPGA}
	\end{itemize}
	\smallskip

% -----
	Software Engineering Internship, Facebook, Menlo Park \hfill \emph{Jun. 2017 - Sep. 2017}
	\begin{itemize}
	\itemsep 0pt
	\item Work in the infrasturcture team to build an offline back test system.
	\item Reprocess all Ads classficiation streams to detect any regression.
	\end{itemize}
	\smallskip
% -----

	% Computer Vision Engineering Internship, DeepGlint, Beijing \hfill \emph{May. 2015 - Jun. 2015}
	% \begin{itemize}
	% \itemsep 0pt
	% \item Work in the computer vision group to develop a "mirror" demo for a skeleton recognition system.
	% \end{itemize}
	% \smallskip

\end{body}
\smallskip
\smallskip


%%%%%%%%%%%%%%%%%%%%%%%%%%%%%%%%%%%%%%%%%%%%%%%%%%%%%%%%%%%%%%%%%%%%%%%%%%%%%%%%
% Skills
\header{Professional \& Personal Skills}

\begin{body}
	\vspace{14pt}
% -----
	Mathematic: Familiar with calculus, linear algebra, probability theory, discrete mathematics, algorithms.\\
	\smallskip
% -----
	Computer Capability: Skilled at C/C++, Python, MATLAB.\\
	\smallskip

	% -----
	Language Proficiency: English: Toefl 114; German: B1 Level(MCER).\\
\end{body}
\smallskip
\smallskip


%%%%%%%%%%%%%%%%%%%%%%%%%%%%%%%%%%%%%%%%%%%%%%%%%%%%%%%%%%%%%%%%%%%%%%%%%%%%%%%%
% Major Courses
% \header{Main Courses}

% \begin{body}
% 	\vspace{14pt}

% 	\begin{itemize} \itemsep -0pt
% 		\item Mining Massive Datasets (Stanford CS246, Coursera)
% 		\item Computer Program Design -- 98, 97 (2012 Fall \& 2013 Spring)
% 		\item Data Structure, Numerical Analysis and Algorithms -- 89 (2013 Fall)
% 		\item Probability and Stochastic Processes(1) -- 100 (2014 Spring)
% 		\item Probability and Stochastic Processes(2) -- A (UW-Madison ECE 730, 2014 Fall)
% 	\end{itemize}

% \end{body}

% \smallskip
% \smallskip

%%%%%%%%%%%%%%%%%%%%%%%%%%%%%%%%%%%%%%%%%%%%%%%%%%%%%%%%%%%%%%%%%%%%%%%%%%%%%%%%
% Experience
\header{Experience}

\begin{body}
	\vspace{14pt}
	% -----
	\textbf{Courses in CS}
	\begin{itemize}
	\itemsep 0pt
	\item Compilers by Alex Aiken, Stanford University
	\item Operating System Engineering, MIT
	\item Programming Languages by Dan Grossman, University of Washington
	\item Machine Learning by Andrew Ng, Stanford University
	\item Algorithms Part I \& II by Robert Sedgewick, Princeton University
	\item Introduction to Computer Science and Programming, MIT
	\item Introduction to Probability, MIT
	\item Advanced Computer Graphics, Tsinghua University
	\item Computer Networks, Tsinghua University
	\item Software Engineering, Tsinghua University
	\item Computer Graphics (5.25/6), ETH Zurich
	\item Computer Vision (5.5/6), ETH Zurich
	\end{itemize}
	\smallskip

	% -----
	Children Education Program Volunteer, \emph{Dream a Dream, Bangalore, India} \hfill \emph{Jul. 2013 - Sept. 2013}\\
	\smallskip

\end{body}

\smallskip
\smallskip



\end{document}