\documentclass[a4paper]{article}
\usepackage{tikz}
\usepackage{palatino}
\usepackage{fullpage}
% \usepackage{CJK}
\usepackage{geometry}
\usepackage[colorlinks = false,
            linkcolor = red,
            ]{hyperref}

% set margins
\leftmargin=0.25in
\oddsidemargin=0.25in
\textwidth=6.0in
\topmargin=-0.25in
\textheight=10in

\raggedright

\def\justifying{%
  \rightskip=0pt
  \spaceskip=0pt
  \xspaceskip=0pt
  \relax
}

\newcommand{\ExternalLink}{%
    \tikz[x=1.2ex, y=1.2ex, baseline=-0.05ex]{% 
        \begin{scope}[x=1ex, y=1ex]
            \clip (-0.1,-0.1) 
                --++ (-0, 1.2) 
                --++ (0.6, 0) 
                --++ (0, -0.6) 
                --++ (0.6, 0) 
                --++ (0, -1);
            \path[draw, 
                line width = 0.5, 
                rounded corners=0.5] 
                (0,0) rectangle (1,1);
        \end{scope}
        \path[draw, line width = 0.5] (0.5, 0.5) 
            -- (1, 1);
        \path[draw, line width = 0.5] (0.6, 1) 
            -- (1, 1) -- (1, 0.6);
        }
    }

\thispagestyle{empty}

\newenvironment{changemargin}[2]{%
  \begin{list}{}{%
    \setlength{\topsep}{0pt}%
    \setlength{\leftmargin}{#1}%
    \setlength{\rightmargin}{#2}%
    \setlength{\listparindent}{\parindent}%
    \setlength{\itemindent}{\parindent}%
    \setlength{\parsep}{\parskip}%
  }%
  \item[]}{\end{list}
}

\newcommand{\lineover}{
	\begin{changemargin}{-0.05in}{-0.05in}
		\vspace*{-8pt}
		\hrulefill \\
		\vspace*{-2pt}
	\end{changemargin}
}

\newcommand{\header}[1]{
	\begin{changemargin}{-0.5in}{-0.5in}
		\scshape{\large \textbf{#1}}\\
  	\lineover
	\end{changemargin}
}

\newcommand{\contact}[4]{
	\begin{changemargin}{-0.5in}{-0.5in}
		\begin{center}
			{\LARGE \scshape \textbf{#1}}\\ \smallskip
			404 Westwood Plaza, EVI 468 – 90095, Los Angeles, CA, USA\\ \smallskip
			{\href{#2}{#2}} ~ \smallskip 
			{\href{#3}{#3}} ~
			{\href{#4}{Google Scholar}}\smallskip
			% {\href{#6}{#6}}\smallskip
		\end{center}
	\end{changemargin}
}

\newenvironment{body} {
	\vspace*{-16pt}
	\begin{changemargin}{-0.5in}{-0.5in}
  }	
	{\end{changemargin}
}	

\newcommand{\school}[4]{
	\textbf{#1} \hfill \emph{#2\\}
	#3\\ 
	#4\\
}

\begin{document}
\contact{Zhengrong Wang}{seanzw@ucla.edu}{https://seanzw.github.io}{https://scholar.google.com/citations?user=h\_GwGfQAAAAJ\&hl=en}

% Bio
\header{Bio}

\begin{body}
	\vspace{14pt}
	\justifying
    \setlength{\parindent}{20pt}%
	
	\indent
	My research aims to build general, automatic and end-to-end near-data acceleration 
	by revolutionizing the orchestration between data and computation throughout the
	entire system.

	\indent
	Conventional von Neumann architectures draw a clear boundary between computation and
	data, in which centralized compute units process the data provided by memory units.
	However, two forces dramatically changed the landscape: 1. Emerging modern applications,
	e.g. large language models, graph nerual networks, recommend systems, etc., scale rapidly
	with the data size (e.g. GPT-4 has 170 trillion parameters), putting extremely high
	pressure on the memory system; 2. The widening gap between the compute and memory
	throughput (known as the memory wall), as well as the upscaling in system size together
	make the data movement an increasingly bottleneck. To continue the performance and
	energy efficiency scaling, my research takes the data as a first-class citizen in
	system design and spans across extensive aspects of data/computation orchestration
	including microarchitecture designs, ISA abstractions, compiler optimizations, and
	codesigning data structures.

	\indent
	I am currently a sixth-year PhD candidate at UCLA. My open source work has been accepted by
	multiple top-tier conferences in computer architecture, including ISCA, MICRO, ASPLOS,
	HPCA, and awarded Best Paper Runner-Ups as well as IEEE Micro Top Pick Honorable
	Mentions. I am also a maintainer of gem5, a widely used cycle accurate simulator in
	computer architecture.

\end{body}

\smallskip
\smallskip

% Education
\header{Education}

\begin{body}
	\vspace{14pt}

% ------
	\textbf{University of California, Los Angeles}, \emph{Department of Computer Science} \hfill Los Angeles, USA \\
Ph.D. Candidate in Computer Science, Advisor: Tony Nowatzki \hfill \emph{Aug. 2018 - Jul. 2024} (Expected){} \\

\vspace{6pt}

% -----
	\textbf{University of California, Los Angeles}, \emph{Department of Computer Science} \hfill Los Angeles, USA \\
Master of Science in Computer Science \hfill \emph{Sep. 2016 - Jul. 2018}{} \\
Thesis: An LLVM-IR Datagraph-Based Simulator for Flexible Design Space Exploration over Accelerator Architectures \\

\vspace{6pt}

% -----
	\textbf{Tsinghua University}, \emph{Department of Electronic Engineering} \hfill Beijing, China \\
Bachlor of Engineering in Electronic Engineering, GPA: 91/100 \hfill \emph{Aug. 2012 - Jul. 2016}{} \\
Thesis: Optimizing Convolutional Neural Network on FPGA under Heterogeneous Computing Framework with OpenCL \\
% Rank: 30/241\\
\vspace{6pt}

% -----
	\textbf{ETH Z\"urich}, \emph{Department of Information Technology} \hfill Z\"urich, Switzerland \\
	Exchange Student, International Academic Program, GPA: 5.50/ 6.00 \hfill \emph{Sept. 2014 - Feb. 2015}{} \\

\end{body}

\smallskip
\smallskip

%%%%%%%%%%%%%%%%%%%%%%%%%%%%%%%%%%%%%%%%%%%%%%%%%%%%%%%%%%%%%%%%%%%%%%%%%%%%%%%%%
% Skills
\header{Publication}

\begin{body}
	\vspace{14pt}
	% -----
    \underline{Inf}inity Stream: Portable and Programmer-Friendly \underline{I}n-/\underline{N}ear-Memory \underline{F}usion \\
	\textbf{Zhengrong Wang}, Christopher Liu, Aman Arora, Lizy John, Tony Nowatzki \\
	\emph{ACM International Conference on Architectural Support for Programming Languages and Operating Systems (ASPLOS)}, 2023, Vancouver, Canada.\\
	\vspace{6pt}
	% -----
    Infinity Stream: Enabling Transparent and Automated In-Memory Computing \\
	\textbf{Zhengrong Wang}, Christopher Liu, Tony Nowatzki \\
    \emph{IEEE Computer Architecture Letters, Vol. 21, No. 2, 2022}.\\
	\vspace{6pt}
	% -----
	OverGen: Improving FPGA Usability through Domain-specific Overlay Generation\\
	Sihao Liu, Jian Weng, Dylan Kupsh, Atefeh Sohrabizadeh, \textbf{Zhengrong Wang}, Licheng Guo, Jiuyang Liu, Maxim Zhulin, Lucheng Zhang, Jason Cong, Tony Nowatzki \\
	\emph{IEEE/ACM International Symposium on Microarchitecture (MICRO)}, 2022, Chicago, USA.\\
	\textbf{Best Paper Runner-Up} \\
	\vspace{6pt}
	% -----
	Near-Stream Computing: General and Transparent Near-Cache Acceleration \\
	\textbf{Zhengrong Wang}, Jian Weng, Sihao Liu, Tony Nowatzki \\
	\emph{IEEE International Symposium on High-Performance Computer Architecture (HPCA)}, 2022, Seoul, South Koera.\\
	\vspace{6pt}
	% -----
	Stream Floating: Enabling Proactive and Decentralized Cache Optimizations \\
	\textbf{Zhengrong Wang}, Jian Weng, Jason Lowe-Power, Jayesh Gaur, Tony Nowatzki \\
	\emph{IEEE International Symposium on High-Performance Computer Architecture (HPCA)}, 2021, Seoul, South Koera.\\
	\textbf{Best Paper Runner-Up} \\
	\vspace{6pt}
	% -----
	DSAGEN: Synthesizing Programmable Spatial Accelerators \\
	Jian Weng, Sihao Liu, Vidushi Dadu, \textbf{Zhengrong Wang}, Preyas Shah, Tony Nowatzki \\
	\emph{ACM International Symposium on Computer Architecture (ISCA)}, 2020, virtual.\\
	\textbf{IEEE Micro Top Picks Honorable Mention}\\
	\vspace{6pt}
	% -----
	A Hybrid Systolic-Dataflow Architecture for Inductive Matrix Algorithms\\
	Jian Weng, Sihao Liu, \textbf{Zhengrong Wang}, Vidush Dadu, Tony Nowatzki \\
	\emph{IEEE International Symposium on High-Performance Computer Architecture (HPCA)}, 2020, San Diego, USA.\\
	\vspace{6pt}
	% -----
	Stream-Based Memory Access Specialization for General Purpose Processors\\
	\textbf{Zhengrong Wang}, Tony Nowatzki \\
	\emph{ACM International Symposium on Computer Architecture (ISCA)}, 2019, Phoenix, USA.\\
	\vspace{6pt}
	% -----
	The Gem5 Simulator: Version 20.0+\\
	Jason Lowe-Power, Abdul Mutaal Ahmad, Ayaz Akram, ..., \textbf{Zhengrong Wang}, et al. \\
	\emph{arXiv:2007.03152v2}, 2020.\\
	\vspace{6pt}
	% -----
	Optimizing Convolutional Neural Network on FPGA under Heterogeneous Computing Framework with OpenCL\\
	\textbf{Zhengrong Wang}, Fei Qiao, Zhen Liu, Yuxiang Shan, Xunyi Zhou, Li Luo, Huazhong Yang \\
	\emph{IEEE Region 10 Conference (TENCON)}, 2016, Singapore.\\
\end{body}

\smallskip
\smallskip

%%%%%%%%%%%%%%%%%%%%%%%%%%%%%%%%%%%%%%%%%%%%%%%%%%%%%%%%%%%%%%%%%%%%%%%%%%%%%%%%
% Awards and Honors

% \newpage
\header{Awards and Honors}

\begin{body}
	\vspace{14pt}
	% -----
	Dissertation Year Fellowship, \emph{UCLA} \hfill{} \emph{June. 2023}\\
	\smallskip
	% -----
	Best Paper Runner-Up (OverGen, in \emph{MICRO '22}), \emph{IEEE} \hfill{} \emph{Oct. 2022}\\
	\smallskip
	% -----
	Best Paper Runner-Up (Stream Floating, in \emph{HPCA '21}), \emph{IEEE} \hfill{} \emph{Feb. 2021}\\
	\smallskip
	% -----
	IEEE Micro Top Picks 2020 Honorable Mention (DSAGEN, in \emph{ISCA '20}), \emph{IEEE} \hfill{} \emph{Jan. 2021}\\
	\smallskip
	% -----
	2021 Dongguan Entrepreneur Scholarship, \emph{Dongguan Entrepreneurs Federation} \hfill{} \emph{Nov. 2021}\\
	\smallskip
	% -----
	Second-class Scholarship for Excellent Freshmen, \emph{Tsinghua University} \hfill{} \emph{Oct. 2012}\\
	\smallskip
	% -----
	Wang Zhaosheng Scholarship for Excellent Student from Dongguan, \emph{Wang Zhaosheng Foundation} \hfill{} \emph{Oct. 2012}\\
	% \smallskip
	% % -----
	% Second Prize in $30^{\mathrm{th}}$ Chinese National Physics Contest (Non-Physical Group A) \hfill{} \emph{Dec. 2013}\\
	% \smallskip
	% % -----
	% Ranked No.5 in National Matriculation Test(Science), Guangdong Province (5/600,000) \hfill{} \emph{Jun. 2012}
\end{body}

\smallskip
\smallskip

%%%%%%%%%%%%%%%%%%%%%%%%%%%%%%%%%%%%%%%%%%%%%%%%%%%%%%%%%%%%%%%%%%%%%%%%%%%%%%%%
% Projects
\header{Open Source Projects \& Infrastructures}

\begin{body}
	\vspace{14pt}

% -----
	\textbf{Stream-Specialized Near-Data Acceleration Framework} \href{https://github.com/PolyArch/gem-forge-framework}{\ExternalLink} \hfill \emph{Jan. 2018 - Present}\\
	First Author \& Maintainer \\
	\begin{itemize}
	\itemsep 0pt
	\item Full-stack implementation of stream-specialized near-data acceleration.
	\item Include LLVM-based compiler transformation and end-to-end simulation in gem5.
	\item Results published in ISCA' 19, HPCA' 21, HPCA '22 and ASPLOS '23. More in submission.
	\end{itemize}
	\smallskip

% -----
	\textbf{Gem5-AVX} \href{https://github.com/seanzw/gem5-avx}{\ExternalLink} \hfill \emph{Jan. 2019 - Present}\\
	First Author \& Maintainer \\
	\begin{itemize}
	\itemsep 0pt
	\item Add AVX-512 support to gem5 simulator, extensively used in research.
	\item Faithfully model the microarchiecture of vectorized instructions, including microops.
	\item Detailed tutorials on how to support new instructions.
	\end{itemize}
	\smallskip
% -----

	\textbf{Gem5 Simulator} \href{https://github.com/gem5/gem5}{\ExternalLink} \hfill \emph{Jan. 2019 - Present}\\
	Committer \& Maintainer \\
	\begin{itemize}
	\itemsep 0pt
	\item Contribute various bug fixes for instruction decoding, microarchitecture deadlock.
	\item Review pull requests.
	\end{itemize}
	\smallskip

% -----

% -----

	% MicroPython on FPGA, Dept. EEE, Imperial College London \hfill \emph{Jul. 2015 - Aug. 2015}
	% \begin{itemize}
	% \itemsep 0pt
	% \item Supervised by Prof. Peter Y. K. Cheung, Head of Dept. EEE.
	% \item Port MicroPython on Altera DE0-Nano-SoC FPGA.
	% \item Build FFT example with DMA.
	% \item Repo: \href{https://github.com/seanzw/MicroPythonFPGA}{https://github.com/seanzw/MicroPythonFPGA}
	% \end{itemize}
	% \smallskip

\end{body}
\smallskip
\smallskip

%%%%%%%%%%%%%%%%%%%%%%%%%%%%%%%%%%%%%%%%%%%%%%%%%%%%%%%%%%%%%%%%%%%%%%%%%%%%%%%%
\header{Professional Experiences}

\begin{body}
	\vspace{14pt}
	\textbf{Nvidia Research} \hfill \emph{Jun. 2022 - Sep. 2022} \\
	Research Scientist, Mentor: Neal Crago, Manager: Steve Keckler
	\begin{itemize}
	\itemsep 0pt
	\item Examine memory bottleneck in GPU for key machine learning kernels.
	\item Build a prototype of an enhanced tensor memory accelerator (TMA).
	\item Evaluted with state of the art point cloud applications.
	\end{itemize}
	\smallskip
% -----
	% Software Engineering Internship, Facebook, Menlo Park \hfill \emph{Jun. 2017 - Sep. 2017}
	% \begin{itemize}
	% \itemsep 0pt
	% \item Work in the infrasturcture team to build an offline back test system.
	% \item Reprocess all Ads classficiation streams to detect any regression.
	% \end{itemize}
	% \smallskip
% -----

	% Computer Vision Engineering Internship, DeepGlint, Beijing \hfill \emph{May. 2015 - Jun. 2015}
	% \begin{itemize}
	% \itemsep 0pt
	% \item Work in the computer vision group to develop a "mirror" demo for a skeleton recognition system.
	% \end{itemize}
	% \smallskip

\end{body}
\smallskip
\smallskip

%%%%%%%%%%%%%%%%%%%%%%%%%%%%%%%%%%%%%%%%%%%%%%%%%%%%%%%%%%%%%%%%%%%%%%%%%%%%%%%%
\newpage
\header{Teaching}

\begin{body}
	\vspace{14pt}
	\textbf{CS33: Introduction to Computer Organization} \hfill \emph{Mar. 2022 - June. 2022} \\
	Teaching Assistant w/ Prof. Glenn Reinman
	\begin{itemize}
	\itemsep 0pt
	\item Lead two-hour discussion every week, lab grading.
	\item Overall evaluation: 8/9.
	\item ``Very knowledgeable, helpful, and encouraging. He ensured that I understood the content."
	\end{itemize}
	\smallskip

\end{body}
\smallskip
\smallskip

%%%%%%%%%%%%%%%%%%%%%%%%%%%%%%%%%%%%%%%%%%%%%%%%%%%%%%%%%%%%%%%%%%%%%%%%%%%%%%%%
\header{Research Mentoring}

\begin{body}
	\vspace{14pt}
	\textbf{Christopher Liu} \href{https://chrisliu.org/}{\ExternalLink} \hfill \emph{CS PhD Student, UCLA} \\
	Develop the compiler support for Infinity Stream (Published in ASPLOS '23)	\hfill \emph{Dec. 2021 - Present} \\
	\smallskip
	\textbf{Nicholas Johnson} \hfill \emph{CS Undergraduate, UCLA} \\
	Improve stream support for high-performance join. \hfill \emph{Jun. 2023 - Sep. 2023} \\
	\smallskip
	\textbf{Arteen Abrishami} \href{arteen1000.github.io}{\ExternalLink} \hfill \emph{CS Undergraduate, UCLA} \\
	Explore near-data acceleration on chiplet architectures. \hfill \emph{Jan. 2023 - Now} \\
	\smallskip
	\textbf{Shyandeep Das} \hfill \emph{EE Master, UCLA} \\
	Support reuse for near-stream computing. Now Graduate Research Architect at ARM. \hfill \emph{Jan. 2023 - Jun. 2023} \\

\end{body}
\smallskip
\smallskip

%%%%%%%%%%%%%%%%%%%%%%%%%%%%%%%%%%%%%%%%%%%%%%%%%%%%%%%%%%%%%%%%%%%%%%%%%%%%%%%%
\header{Community Services}

\begin{body}
	\vspace{14pt}
	\textbf{IEEE Transactions on Computers} Reviewer \hfill \emph{Jul. 2023} \\
	\smallskip
	\textbf{ACM Transactions on Architecture and Code Optimization} Reviewer \hfill \emph{Jul. 2023} \\
	\smallskip
	\textbf{ISCA '23 uArch Workshop} \hfill \emph{Jun. 2023} \\
	Mentor for 4 talented undergraduate computer architects: \\
	Shujuan Chen (UC Davis), Viansa Schmulbach (UC Berkeley), \\
	Eden Alem (Washington University in St. Louis), and Evan Cheng (Stanford). \\
	\smallskip
	\textbf{HPCA '21} Student Reviewer \hfill \emph{Aug. 2020}\\

\end{body}
\smallskip
\smallskip


%%%%%%%%%%%%%%%%%%%%%%%%%%%%%%%%%%%%%%%%%%%%%%%%%%%%%%%%%%%%%%%%%%%%%%%%%%%%%%%%
% Projects
% \header{Selected Projects \& Internships}

% \begin{body}
% 	\vspace{14pt}

% 	Yart-cpp, yet another ray tracer in C++ \hfill \emph{Sep. 2015 - Mar. 2016}
% 	\begin{itemize}
% 	\itemsep 0pt
% 	\item Implement direct lighting and bidirectional path tracing.
% 	\item Lambertian, specular, refraction and Cook-Torrance BRDF are supported.
% 	\item Use OC-Tree to accelerate complex mesh.
% 	\item Repo: \href{https://github.com/seanzw/yart-cpp}{https://github.com/seanzw/yart-cpp}
% 	\end{itemize}
% 	\smallskip

% % -----
% % -----

% 	Light C-Compiler in C\# \hfill \emph{Dec. 2015 - Mar. 2016}
% 	\begin{itemize}
% 	\itemsep 0pt
% 	\item Support C99 standard and generate x86 assembly.
% 	\item Handwritten regular expression, NFA scanner generator and parser combinator library.
% 	\item Scanner generated by handwritten scanner generator.
% 	\item Parser generated by handwritten parser combinator library.
% 	\item Semantic analysis that checks types and symbols;
% 	\item Code generation to x86 assembly on Windows (can be directly used by clang assembler)
% 	\item Repo: \href{https://github.com/seanzw/lcc}{https://github.com/seanzw/lcc}
% 	\end{itemize}
% 	\smallskip

% -----
% -----

	% OpenCL@FPGA (Undergraduate Thesis) \hfill \emph{Sep. 2015 - Jun. 2016}
	% \begin{itemize}
	% \itemsep 0pt
	% \item Supervised by Assoc. Prof. Fei Qiao, Tsinghua University
	% \item Use OpenCL to implement CNN on Xilinx Alpha Data FPGA, and accelerate with pipeline.
	% \item Paper on TENCON 16: Optimizing Convolutional Neural Network on FPGA under Heterogeneous Computing Framework with OpenCL
	% \end{itemize}
	% \smallskip

% \end{body}
% \smallskip
% \smallskip


% %%%%%%%%%%%%%%%%%%%%%%%%%%%%%%%%%%%%%%%%%%%%%%%%%%%%%%%%%%%%%%%%%%%%%%%%%%%%%%%%
% % Skills
% \header{Professional \& Personal Skills}

% \begin{body}
% 	\vspace{14pt}
% % -----
% 	Mathematic: Familiar with calculus, linear algebra, probability theory, discrete mathematics, algorithms.\\
% 	\smallskip
% % -----
% 	Computer Capability: Skilled at C/C++, Python, MATLAB.\\
% 	\smallskip

% 	% -----
% 	Language Proficiency: English: Toefl 114; German: B1 Level(MCER).\\
% \end{body}
% \smallskip
% \smallskip


% %%%%%%%%%%%%%%%%%%%%%%%%%%%%%%%%%%%%%%%%%%%%%%%%%%%%%%%%%%%%%%%%%%%%%%%%%%%%%%%%
% % Major Courses
% % \header{Main Courses}

% % \begin{body}
% % 	\vspace{14pt}

% % 	\begin{itemize} \itemsep -0pt
% % 		\item Mining Massive Datasets (Stanford CS246, Coursera)
% % 		\item Computer Program Design -- 98, 97 (2012 Fall \& 2013 Spring)
% % 		\item Data Structure, Numerical Analysis and Algorithms -- 89 (2013 Fall)
% % 		\item Probability and Stochastic Processes(1) -- 100 (2014 Spring)
% % 		\item Probability and Stochastic Processes(2) -- A (UW-Madison ECE 730, 2014 Fall)
% % 	\end{itemize}

% % \end{body}

% % \smallskip
% % \smallskip

% %%%%%%%%%%%%%%%%%%%%%%%%%%%%%%%%%%%%%%%%%%%%%%%%%%%%%%%%%%%%%%%%%%%%%%%%%%%%%%%%
% % Experience
% \header{Experience}

% \begin{body}
% 	\vspace{14pt}
% 	% -----
% 	\textbf{Courses in CS}
% 	\begin{itemize}
% 	\itemsep 0pt
% 	\item Compilers by Alex Aiken, Stanford University
% 	\item Operating System Engineering, MIT
% 	\item Programming Languages by Dan Grossman, University of Washington
% 	\item Machine Learning by Andrew Ng, Stanford University
% 	\item Algorithms Part I \& II by Robert Sedgewick, Princeton University
% 	\item Introduction to Computer Science and Programming, MIT
% 	\item Introduction to Probability, MIT
% 	\item Advanced Computer Graphics, Tsinghua University
% 	\item Computer Networks, Tsinghua University
% 	\item Software Engineering, Tsinghua University
% 	\item Computer Graphics (5.25/6), ETH Zurich
% 	\item Computer Vision (5.5/6), ETH Zurich
% 	\end{itemize}
% 	\smallskip

% 	% -----
% 	Children Education Program Volunteer, \emph{Dream a Dream, Bangalore, India} \hfill \emph{Jul. 2013 - Sept. 2013}\\
% 	\smallskip

% \end{body}

\smallskip
\smallskip



\end{document}